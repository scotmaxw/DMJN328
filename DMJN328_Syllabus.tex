\documentclass{article}
\usepackage{csquotes}
% !BIB TS-program = biber
\usepackage[backend=biber, citestyle=authoryear]{biblatex}
\addbibresource{//Users/Simon/OneDrive - Wilfrid Laurier University/Courses/DMJN328/DMJN328.bib}
\usepackage{url}
\title{DMJN328}
\author{Simon J. Kiss\\\\Term\\
Wilfrid Laurier University - Brantford\\
Day and Time\\Office: RCW 312 \\
Phone: 519-756-8228\\
E-mail: please contact me via mylearningspace\\
Office Hours: Day and Time}
\date{}
\begin{document}
\maketitle
\begin{center}

\end{center}
\section*{Course Description}

\subsection*{Pep Talk}
Learning R can be tough; Lord knows I struggled to learn it. But learning anything is always hard. But when you have really learned it. Wow, there is no better feeling in the world.  And once you have learned even just a few basic things in R you - and others - will be absolutely amazed at what you can accomplish.

One of the leaders in developing R packages including almost all of the packages you will be using in this course, Hadley Wickham had this to say about frustration and learning to code and R. 
\begin{quote}
 
It's easy when you start out programming to get really frustrated and think, ''Oh it?s me, I?m really stupid,'' or, ''I?m not made out to program.'' But, that is absolutely not the case. Everyone gets frustrated. I still get frustrated occasionally when writing R code. It?s just a natural part of programming. So, it happens to everyone and gets less and less over time. Don?t blame yourself. Just take a break, do something fun, and then come back and try again later.
\end{quote}

There will be moments where you want to bash your head against the desk.  Don\\'t. Ask me for help any way (e.g. personally, e-mail, via Twitter, Slack, mylearningspace). Take a break. Walk around. Go get a drink. Watch cat videos on YouTube for a few minutes then just come back and take a break.

You. can. do. this...

And when you're done, you will be blown away....

\subsection*{Learning Outcomes}
\begin{enumerate}
\item \end{enumerate}
 
\section*{Course Calendar And Readings}
Monday, January 6, 2020\\
Introduction, Introuction to R, RStudio and Github\\
Discussion of \emph{The Joy of Stats}\\
Wednesday, January 8, 2020\\
Read Chapter 1 and 2 of \cite{healy_data_2018}\\
Class Complete `Week 1: Dataviz`
Monday, January 13, 2020\\

Wednesday, January 15, 2020
Monday, January 20, 2020
Wednesday, January 22, 2020
Monday, January 27, 2020
Wednesday, January 29, 2020
Monday, February 3, 2020
Wednesday, February 5, 2020
Monday, February 10, 2020
Wednesday, February 12, 2020
Monday, February 17, 2020
Wednesday, February 19, 2020
Monday, February 24, 2020
Wednesday, February 26, 2020
Monday, March 2, 2020
Wednesday, March 4, 2020
Monday, March 9, 2020
Wednesday, March 11, 2020
Monday, March 16, 2020
Wednesday, March 18, 2020
Monday, March 23, 2020
Wednesday, March 25, 2020
Monday, March 30, 2020
Wednesday, April 1, 2020


\section*{Course Readings}
available from the bookstore. 
\section*{Software Requirements}
\begin{enumerate}
\item Please come with an account created at http://www.github.com. Github is aq leading platform for collaborating on software development projects as well as sharing data and code with the public.  

\end{enumerate}

\section*{Assignments:}
\begin{table}[ht]
\center
\begin{tabular}{lrr}
Assignment & Weight & Due Date\\
\hline
&	&\\
&			&\\
	&		&		\\
&  &\\
& & \\

\end{tabular}
\end{table}

\subsection*{Assignment Descriptions}

\begin{enumerate}
\item 
\end{enumerate}



\section*{Late Submissions}

\section*{E-mail contact:}
Please contact me on mylearningspace.

\section*{Academic Integrity}
Wilfrid Laurier University uses software that can check for plagiarism. if requested to do so by the instructor, students are required to submit their written work in electronic form and have it checked for plagiarism.

Laurier is committed to a culture of integrity within and beyond the classroom. This culture values trustworthiness (i.e., honesty, integrity, reliability), fairness, caring, respect, responsibility and citizenship. Together, we have a shared responsibility to uphold this culture in our academic and nonacademic behaviour. The University has a defined policy with respect to academic misconduct. As a Laurier student you are responsible for familiarizing yourself with this policy and the accompanying penalty guidelines, some of which may appear on your transcript if there is a finding of misconduct. The relevant policy can be found at Laurier's academic integrity website along with resources to educate and support you in upholding a culture of integrity. Ignorance is not a defense.



\section*{Accessible Learning}
Students with special needs are advised to contact Laurier's Accessible Learning Office for information regarding its services and resources. They are also encouraged to review the Calendar for information regarding all services available on campus. 

\section*{General Information}
\begin{enumerate}


\item Academic Calendars: Students are encouraged to review the Academic Calendar for information regarding all important dates, deadlines, and services available on campus.

\item Classroom Use of Electronic Devices: Students are free to use electronic devices - except smart phones - for study and learning purposes only. 

\item Final Examinations: Students are strongly urged not to make any commitments (i.e., vacation) during the examination period. Students are required to be available for examinations during the examination periods of all terms in which they register. 

\subsection*{Brantford Resources:}

\begin{itemize}


\item Brantford Student Food Bank (\url{http://yourstudentsunion.ca/service/food-bank/}): All students are eligible to use this service to ensure they?re eating healthy when overwhelmed, stressed or financially strained. Anonymously request a package online 24-7. All dietary restrictions accommodated.

\item Brantford Foot Patrol (\url{https://students.wlu.ca/wellness-and-recreation/safety/foot-patrol.html}): 519-751-PTRL (7875). A volunteer operated safe-walk program, available Fall and Winter, Monday through Thursday from 6:30 pm to 1 am; Friday through Sunday 6:30 pm to 11 pm. Teams of two are assigned to escort students to and from campus by foot or by van.

\item Brantford Wellness Centre (\url{https://students.wlu.ca/wellness-and-recreation/safety/foot-patrol.html}): 519-756-8228, x5803. Students have access to support for all their physical, emotional, and mental health needs at the Wellness Centre. Location: Student Centre, 2nd floor. Hours: 8:30 am to 4:15 pm Monday through Friday. After hours crisis support available 24/7. Call 1-884-437-3247 (HERE247). 

Multi-campus Resource:
\item 
Good2Talk is a postsecondary school helpline that provides free, professional and confidential counselling support for students in Ontario. Call 1-866-925-5454 or through 2-1-1. Available 24-7.
\end{itemize}
\end{enumerate}

\end{document}