\documentclass{article}
\usepackage{csquotes}
% !BIB TS-program = biber
% !BIB program = biber
\usepackage[backend=biber, citestyle=authoryear]{biblatex}
\addbibresource{/Users/skiss/OneDrive - Wilfrid Laurier University/Courses/DMJN328/DMJN328.bib}
\usepackage{url}
\usepackage{enumitem}
\usepackage{hyperref}
\setlist[enumerate]{nosep}
\title{DMJN328: Quantitative Research Methods for Journalists}
\author{Simon J. Kiss\\\\Winter 2020\\
Wilfrid Laurier University - Brantford\\
MW 10-11:30 OD211\\Office: RCW 312 \\
Phone: 519-756-8228\\
E-mail: please contact me via mylearningspace\\
Office Hours: Mondays 11:30 a.m. - 12:30 p.m.}
\date{}
\begin{document}
\maketitle
\begin{center}

\end{center}
\section*{Course Description}
This course builds on material in DMJN208, developing students’ facility with quantitative research methods used in journalism and media research. Students learn to work with and critically assess quantitative information, with a particular emphasis on polls and scientific studies. The course covers basic and intermediate statistical concepts and calculations. Potential topics include survey research design and interpretation, causation, the central limit theorem, standard error, statistical significance and confidence intervals.

\subsection*{Pep Talk}
Learning R can be tough; Lord knows I struggled to learn it. But learning anything is always hard. But when you have really learned it. Wow, there is no better feeling in the world.  And once you have learned even just a few basic things in R you - and others - will be absolutely amazed at what you can accomplish.

One of the leaders in developing R, Hadley Wickham, had this to say about frustration and learning to code and R: 

\begin{quote}
 It's easy when you start out programming to get really frustrated and think, ''Oh it's me, I'm really stupid,'' or, ''I'm not made out to program.'' But, that is absolutely not the case. Everyone gets frustrated. I still get frustrated occasionally when writing R code. It's just a natural part of programming. So, it happens to everyone and gets less and less over time. Don't blame yourself. Just take a break, do something fun, and then come back and try again later.
\end{quote}

There will be moments where you want to bash your head against the desk.  Don't. Ask me for help any way (e.g. personally, e-mail, mylearningspace). Take a break. Walk around. Go get a drink. Watch cat videos on YouTube for a few minutes then just come back and take a break.

You. can. do. this...

And when you're done, you will be blown away....

\subsection*{Learning Outcomes}
At the end of this course will be able to:
\begin{enumerate}
\item understand basic principles of data visualization
\item import successfully a variety of datasets into R
\item to manipulate different types of variables found in modern datasets in R
\item to produce and modify basic charts and graphs in R
\item understand basic statistical concepts such as measures of central tendency, variation and inference. 
\end{enumerate}

\section*{Course Calendar And Readings}
\textbf{Monday, January 6, 2020}\\
\emph{Introduction}
\begin{enumerate}
\item Discussion of \emph{The Joy of Stats}
\item Introduction to R and RStudio
\item Introductdion to GitHub
\end{enumerate}
\bigskip
\textbf{Wednesday, January 8, 2020\\}
\emph{Introduction}\\
\emph{Before} class read Ch. 2 of \textcite{healy_data_2018}\\
\emph{In Class} work through code in Ch. 2 of \textcite{healy_data_2018}\\\\
\textbf{Monday, January 13,2020\\}
\emph{Introduction to Data Visualization in R\\}
\emph{Before class} read Chapter 3 of \textcite{healy_data_2018}\\
\emph{In class} work through code in Ch. 3 of \textcite{healy_data_2018}\\\\
\textbf{Wednesday, January 15, 2020\\}
\emph{Introduction to R For Journalists}\\
\emph{Before class} watch and work through code in: 
\begin{enumerate}
\item \href{http://learn.r-journalism.com/en/how_to_use_r/}{How to Use R}
\begin{enumerate}
\item \href{http://learn.r-journalism.com/en/how_to_use_r/tour_rstudio/rstudio-tour/}{Rstudio Guide}Module 1, Part 1(7:36)
\item \href{http://learn.r-journalism.com/en/how_to_use_r/intro_to_r/intro-to-r/}{Introduction to R} Module 1, Part 2 (15:05) and Part 3 (18:51)
\item \href{http://learn.r-journalism.com/en/how_to_use_r/initial_exploration/data-structures/}{Data structures in R} Module 1, Part 4 (19:11)
\item Setting Up For This Class (5:13)
\item R Basics
\end{enumerate}

\end{enumerate}
\bigskip
\emph{In class} conduct exercises at the end of: 
    
\begin{enumerate}
\item How to use R
\begin{enumerate}

\item \href{http://code.r-journalism.com/chapter-1/#section-intro-to-r}{Introduction to R}
\item \href{http://code.r-journalism.com/chapter-1/#section-data-structures-in-r}{Data Structures}
\end{enumerate}
\end{enumerate}
\bigskip
\textbf{Monday, January 20, 2020\\}
\emph{Introduction to R For Journalists}\\
Before class watch and work through code in:
\begin{enumerate}
\item \href{https://youtu.be/IrCpddfUVsY}{Importing/Exporting Data} Module 1 Video 6 (12:55) \underline{(Optional) }
\begin{enumerate}
\item CSV Files 
\item \href{https://youtu.be/B5iKikPvdBk}{Excel Files} Module 1 Video 8 (11:57)
\item \href{https://youtu.be/q3p6_v_6g9c}{SPSS Files} Module 1 Video 12 (13:55)
\end{enumerate}
\end{enumerate}
\bigskip
\emph{In class} work through exercises at the end of:
\begin{enumerate}
\item Importing/Exporting Data
\begin{enumerate}
\item \href{http://code.r-journalism.com/chapter-2/#section-csvs}{CSV Files}
\item \href{http://code.r-journalism.com/chapter-2/#section-excel}{Excel Files}
\end{enumerate}	
\end{enumerate}
\bigskip
\textbf{Wednesday, January 22, 2020\\}
Where to search for data?\\
\begin{enumerate}
\item Statistics Canada data
\item Public Opinion Data
\item Municipal data 
\item Open government sites 
\end{enumerate}
\bigskip
\textbf{Monday, January 27, 2020\\}
\emph{R for Journalists}\\
Before class watch and work through code in: 
\begin{enumerate}
\item \href{http://learn.r-journalism.com/en/wrangling/}{Wrangling data}Module 2, Part 1 (2:42)
\begin{enumerate}
\item \href{http://learn.r-journalism.com/en/wrangling/dplyr/dplyr/}{Transforming and analyzing data} Module 2, Part 2 (21:27), 
\item \href{http://learn.r-journalism.com/en/wrangling/dplyr/dplyr/}{Transforming and analyzing data} Module 2, Part 3 (14:56)
\item \href{http://learn.r-journalism.com/en/wrangling/dplyr/dplyr/}{Transforming and analyzing data} Module 2, Part 4 (20:21)
\end{enumerate}
\end{enumerate}
\bigskip
In class conduct exercises for
\begin{enumerate}
\item Wrangling data
\begin{enumerate}
\item \href{http://code.r-journalism.com/chapter-3/#section-transforming-and-analyzing-data}{Transforming and analyzing data}

\end{enumerate}

\end{enumerate}
\bigskip
\textbf{Monday, February 3, 2020\\}
\emph{R for Journalists}\\

\begin{enumerate}
\item Wrangling Data
\begin{enumerate}
\item \href{http://learn.r-journalism.com/en/wrangling/tidyr_joins/tidyr-joins/}{Tidying and joining data} Module 2, Part 5 (22:03)
\item \href{http://learn.r-journalism.com/en/wrangling/tidyr_joins/tidyr-joins/}{Tidying and joining data} Module 2, Part 6 (13:46)
\end{enumerate}

\end{enumerate}


\bigskip
In class work through:
\begin{enumerate}
\item Wrangling Data 
\begin{enumerate}
\item \href{http://code.r-journalism.com/chapter-3/#section-tidying-and-joining-data}{Tidying and joining data}
\end{enumerate}
\end{enumerate}
\bigskip
\textbf{Wednesday, February 5, 2020\\}
\emph{R For Journalists}\\
Before class watch and work through code in:
\begin{enumerate}
\item Wrangling Data
\begin{enumerate}
\item \href{http://learn.r-journalism.com/en/wrangling/strings/strings/} Module 2, part 8
\item \href{http://learn.r-journalism.com/en/wrangling/dates/dates/}{Dealing with Dates} Module 2, part 9
\end{enumerate}
\end{enumerate}
\bigskip
In class work through exercises for:
\begin{enumerate}
\item Wrangling data
\begin{enumerate}
\item \href{http://code.r-journalism.com/chapter-3/#section-handling-strings}{Handling Strings}
\item \href{http://code.r-journalism.com/chapter-3/#section-dealing-with-dates}{Dealing with dates}
\end{enumerate}
\end{enumerate}
\bigskip
\textbf{Monday, February 10, 2020\\}
\emph{R For Journalists}\\
Before class watch and work through code in:
\begin{enumerate}
\item Wrangling Data
\begin{enumerate}
\item \href{https://youtu.be/yt9x6PNYvlw}{Dealing with Dates}
\end{enumerate}

\end{enumerate}
\bigskip
In class work through exercises in:
\begin{enumerate}
\item Wrangling Data
\begin{enumerate}
\item \href{http://code.r-journalism.com/chapter-3/#section-dealing-with-dates}(Dealing with Dates)
\end{enumerate}
\end{enumerate}
\bigskip
\textbf{Wednesday, February 12, 2020\\
}
In class work and support on Assignment 2 \\
\textbf{Monday, February 17-19, 2020\\}
Reading Week - No classes\\\\
\textbf{Monday, February 24, 2020}\\
\emph{R For Journalists}\\
Before class watch
\begin{enumerate}
\item Visualizing data \href{https://youtu.be/8VW-APX_5a0}{Module 3, Video, Part 1}(5:00)
\begin{enumerate}
\item Charts with ggplot2 \href{https://youtu.be/ZBewoHKyMcc}{Module 3, Video, Part 2}(18:50)
\item Charts with ggplot2 \href{https://youtu.be/x4OMSY2kz8M}{Module 3, Video, Part 3}(20:01)
\item Charts with ggplot2 \href{https://youtu.be/xjnj-rJwd6A}{Module 3, Video, Part 4}(12:31)
\end{enumerate}
\end{enumerate}
\bigskip
In class, work through exercises 
\begin{enumerate}
\item Visualizing Data
\begin{enumerate}
\item \href{http://code.r-journalism.com/chapter-4/#section-ggplot2}{Charts with ggplot2}
\item \href{http://code.r-journalism.com/chapter-4/#section-customizing-charts}{Customizing charts}
\end{enumerate}
\end{enumerate}
\bigskip
\textbf{Wednesday, February 26, 2020}\\
\emph{R For Journalists}\\
Review, Project Work\\\\
\textbf{Monday, March 2, 2020\\}
\emph{R For Journalists}\\
Before class watch:

\begin{enumerate}
\item Spatial analysis \href{https://youtu.be/iF050DaUxC4}{Module 4, Part1}
\begin{enumerate}
\item Static Maps \href{https://youtu.be/-udJxD9DkA8}{Module 4, Part 2} (20:54)
\item Static Maps \href{https://youtu.be/fyt7UqYEESs}{Module 4, Part 3}(29:03)
\end{enumerate}

\end{enumerate}
\bigskip 
In class work through exercises: 
\begin{enumerate}
\item \href{http://code.r-journalism.com/chapter-5/#section-shape-files}{Shape Files}
\item \href{http://code.r-journalism.com/chapter-5/#section-small-multiples}{Small Multiples}
\end{enumerate}
\bigskip
\textbf{Wednesday, March 4, 2020\\}
Review class\\\\
\textbf{Monday, March 9, 2020\\}
\emph{Advanced statistics for Journalists}\\
Why quantify?\\
Levels of Measurement\\\\
\textbf{Wednesday, March 11, 2020\\}\\
\emph{Advanced statistics for Journalists}\\
Measures of Central Tendency\\\\
\textbf{Monday, March 16, 2020\\}
\emph{Advanced statistics for Journalists}\\
Sampling Error, Central Limit Theorem, Uncertainty\\\\
\textbf{Wednesday, March 18, 2020\\}
\emph{Advanced statistics for Journalists}\\
Causation, Research Design\\\\
\textbf{Monday, March 23, 2020\\}
\emph{Advanced statistics for Journalists}\\
Statistical Significance\\\\
\textbf{Wednesday, March 25, 2020\\}
\emph{Advanced statistics for Journalists}\\
Probability and the Crisis in Science. \\\\
\textbf{Monday, March 30, 2020\\}
Review, Wrap up\\\\
\textbf{Wednesday, April 1, 2020\\}
FIN!\\

\section*{Course Readings}
Please purchase 
\fullcite{healy_data_2018} and \fullcite{stray_curious_2016} (available as a coursepack in the bookstore)
\section*{Software Requirements}
We will be using three different software packages in this course: R, RStudio and GitHub Desktop. Each of these software packages will be installed for your use on all the computers in OD211. At a basic level, you should be able to complete the course without bringing any personal technology. However, you may find it useful to install this software on your personal computer so that you can work on the course material at your own pace. 

\subsection*{R}
\begin{enumerate}
\item Visit \url{https://cran.r-project.org/index.html}
\item Select Download R for (Mac) OS X or Download R for Windows
\item Select the latest package of R for download and install as you would any regular software. For macs, this involves double-clicking the .pkg file that is downloaded and walking through the steps in the dialogue menu. 
\end{enumerate}

\subsection*{RStudio }

\begin{enumerate}
\item Visit \url{https://rstudio.com/products/rstudio/download/}
\item Select the Free Desktop version; it should direct you to a download screen appropriate to your operating system. 
\end{enumerate}

\subsection*{GitHub Desktop}
\begin{enumerate}
\item Visit \url{https://desktop.github.com/}
\item Select Download for MacOs or Download for Windows (as appropriate)
\item Install
\end{enumerate}



\begin{enumerate}

\item Please come with an account created at \href{http://www.github.com}. Github is a leading platform for collaborating on software development projects as well as sharing data and code with the public.  
\item 

\end{enumerate}

\section*{Assignments:}
\begin{table}[ht]
\center
\begin{tabular}{lrr}
Assignment & Weight & Due Date\\
\hline
News Data Visualization Presentation& 10\%	& TBA\\
Graph of data 			&25 \% & TBA\\
Map 		& 25\% 		& TBA \\
Final Exam & 40\%  & TBA\\


\end{tabular}
\end{table}

\subsection*{Assignment Descriptions}
\subsubsection*{News Data Journalism Visualization Presentation}
In this assignment students will present a recent example of some kind of data visualization for a news media outlet, either online or in print. Student will present the data visualization to the class and discuss it with the class (pass/fail). 

\subsubsection*{Graph of Data}
In this assignment, students will present one original graph of some kind of data that they find newsworthy. Part of the assignment will include finding the data, importing it into R, manipulating it fairly and as necessary and producing a compelling, attractive graph that communicates a newsworthy pattern. 

\subsubsection*{Map}
In this assignment, students will present one original map of some kind kind of data that they find newsworthy. Part of the assignment will include finding the data, importing it into R, manipulating it fairly and as necessary and producing a compelling, attractive map that communicates a newsworthy patterns. 

\subsubsection*{Final Exam}
A final exam will be conducted that covers basic statistical concepts covered in the course and the text \textcite{stray_curious_2016} (see Course Readings).

\section*{Late Submissions}

\section*{E-mail contact:}
Please contact me on mylearningspace.

\section*{Academic Integrity}
Wilfrid Laurier University uses software that can check for plagiarism. if requested to do so by the instructor, students are required to submit their written work in electronic form and have it checked for plagiarism.

Laurier is committed to a culture of integrity within and beyond the classroom. This culture values trustworthiness (i.e., honesty, integrity, reliability), fairness, caring, respect, responsibility and citizenship. Together, we have a shared responsibility to uphold this culture in our academic and nonacademic behaviour. The University has a defined policy with respect to academic misconduct. As a Laurier student you are responsible for familiarizing yourself with this policy and the accompanying penalty guidelines, some of which may appear on your transcript if there is a finding of misconduct. The relevant policy can be found at Laurier's academic integrity website along with resources to educate and support you in upholding a culture of integrity. Ignorance is not a defense.



\section*{Accessible Learning}
Students with special needs are advised to contact Laurier's Accessible Learning Office for information regarding its services and resources. They are also encouraged to review the Calendar for information regarding all services available on campus. 

\section*{General Information}
\begin{enumerate}


\item Academic Calendars: Students are encouraged to review the Academic Calendar for information regarding all important dates, deadlines, and services available on campus.

\item Classroom Use of Electronic Devices: Students are free to use electronic devices - except smart phones - for study and learning purposes only. 

\item Final Examinations: Students are strongly urged not to make any commitments (i.e., vacation) during the examination period. Students are required to be available for examinations during the examination periods of all terms in which they register. 

\subsection*{Brantford Resources:}

\begin{itemize}


\item Brantford Student Food Bank (\url{http://yourstudentsunion.ca/service/food-bank/}): All students are eligible to use this service to ensure they?re eating healthy when overwhelmed, stressed or financially strained. Anonymously request a package online 24-7. All dietary restrictions accommodated.

\item Brantford Foot Patrol (\url{https://students.wlu.ca/wellness-and-recreation/safety/foot-patrol.html}): 519-751-PTRL (7875). A volunteer operated safe-walk program, available Fall and Winter, Monday through Thursday from 6:30 pm to 1 am; Friday through Sunday 6:30 pm to 11 pm. Teams of two are assigned to escort students to and from campus by foot or by van.

\item Brantford Wellness Centre (\url{https://students.wlu.ca/wellness-and-recreation/safety/foot-patrol.html}): 519-756-8228, x5803. Students have access to support for all their physical, emotional, and mental health needs at the Wellness Centre. Location: Student Centre, 2nd floor. Hours: 8:30 am to 4:15 pm Monday through Friday. After hours crisis support available 24/7. Call 1-884-437-3247 (HERE247). 

Multi-campus Resource:
\item 
Good2Talk is a postsecondary school helpline that provides free, professional and confidential counselling support for students in Ontario. Call 1-866-925-5454 or through 2-1-1. Available 24-7.
\end{itemize}
\end{enumerate}

\end{document}